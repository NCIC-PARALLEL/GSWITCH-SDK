\section{Submission Guidelines}

This document is intended to serve as a sample for submissions to ISCA 2019. It is heavily derived from previous conferences, in particular MICRO 2015, HPCA 2017, ISCA 2017, and ISCA 2018.

We provide some guidelines that authors should follow when submitting papers to the conference. \textbf{This format is derived from the ACM \texttt{sig-alternate.cls} file, with the major difference that it is 10pt Times font}.

In an effort to respect the efforts of reviewers and in the interest of fairness to all prospective authors, we request that all submissions follow the formatting and submission rules detailed below. Submissions that (grossly) violate these instructions may not be reviewed, at the discretion of the Program Chair, in order to maintain a review process that is fair to all potential authors.

\vspace{1ex}In a nutshell:

\begin{itemize}
\item Paper must be submitted in printable PDF format.
\item Text must be minimum 10pt Times font.
\item Papers must be at most 11 pages, not including references, in two-column format.
\item References must include all authors to facilitate the reviewing process (no \emph{et al.})
\item There is no page limit for references.
\end{itemize}

\section{Paper Formatting}

Papers must be submitted in printable PDF format and should contain a maximum of 11 pages of single-spaced, two-column text, not including references. You may include any number of pages for references, but see below for more instructions.

\begin{itemize}

\item If you are using \LaTeX~\cite{lamport94}to typeset your paper, then we strongly recommend that you use the template provided. \textbf{Please set your document to Times font \\ (\texttt{\textbackslash usepackage\{mathptmx\}}) and do not play with interline spacing}.

\item If you are using a different software package to typeset your paper, then please adhere to the guidelines mentioned in the table below. \textbf{You must use 10pt Times font or larger}.

\end{itemize}

\begin{table}[h]
\caption{Formatting guidelines.}

\begin{tabular}{ll}
\hline
Field &
Value \\
\hline
Page limit &
11 pages w/o refs. \\
Paper size &
US Letter 8.5x11in \\
Top margin &
1in \\
Bottom margin &
1in \\
Left margin &
0.75in \\
Right margin &
0.75in \\
Body &
2-col., single-spaced \\
Separation between columns &
0.25in \\
Body font &
10pt Times \\
Abstract font &
10pt Times \\
Section heading font &
12pt, bold \\
Subsection heading font &
10pt, bold \\
Caption font &
9pt (minimum), bold \\
References &
8pt, no page limit \\
\hline
\end{tabular}

\end{table}

Please ensure that you include page numbers with your submission. This makes it easier for the reviewers to refer to different parts of your paper when they provide comments. Please ensure that your submission has a banner at the top of the title page which contains the submission number and a notice of confidentiality.

\section{Content}

\subsection{Author List}  Reviewing will be double-blind; therefore, please do not include any author names on any submitted documents except in the space provided on the submission form. You must also ensure that the metadata included in the PDF does not give away the authors. If you are improving upon your prior work, refer to your prior work in the third person and include a full citation for the work in the bibliography. For example, if you are building on your own prior work in the papers~\cite{nicepaper1,nicepaper2,nicepaper3}, you would say something like: ``While prior work did X, Y, and Z~\cite{nicepaper1,nicepaper2,nicepaper3}, this paper additionally does W, and is therefore much better.'' Do not omit or anonymize references for blind review. There is one exception to this for your own prior work that appeared in IEEE CAL, workshops without archived proceedings, etc., as discussed later in this document.

Recall that, per IEEE authorship guidelines, it is not acceptable to award honorary authorship or gift authorship. Please keep these guidelines in mind while determining the author list of your paper. Also please note that addition/removal of authors once the paper is accepted will have to be approved by the Program Chair.

\subsection{Figures and Tables}

Ensure that the figures and tables are legible. Please also ensure that you refer to your figures in the main text. Many reviewers print the papers in gray-scale. Therefore, if you use colors for your figures, ensure that the different colors are highly distinguishable in gray-scale.

\subsection{References}

There is no length limit for references. Each reference must explicitly list all authors of the paper. Papers not meeting this requirement will be rejected.

\section{Conflicts of Interest}

Authors must register all their conflicts on the paper submission site. Conflicts are needed to ensure appropriate assignment of reviewers. If a paper is found to have an undeclared conflict that causes a problem OR if a paper is found to declare false conflicts in order to abuse or game the review system, the paper may be rejected.

Please declare a conflict of interest with the following people for any author of your paper.
A conflict occurs in the following cases:
\begin{enumerate}
\item Between advisor and advisee forever. 
\item Between family members forever. 
\item Between people who have collaborated in the last 5 years. This collaboration can consist of a joint research or development project, a joint paper, or when there is direct funding from the potential reviewer (as opposed to company funding) to an author of the paper. Co-participation in professional activities, such as tutorials or studies, is not cause for conflict. When in doubt, the author should check with the Program Chair. 
\item Between people from same institution or who were in the same institution in the last 5 years. 
\item Between people whose relationship prevents the reviewer from being objective in his/her assessment
\end{enumerate}


``Service'' collaborations, such as co-authoring a report for a professional organization, serving on a program committee, or co-presenting tutorials, do not themselves create a conflict of interest. Co-authoring a paper that is a compendium of various projects with no true collaboration among the projects does not constitute a conflict among the authors of the different projects.

We hope to draw most reviewers from the PC and the ERC, but others from the community may also write reviews. Please declare all your conflicts (not just restricted to the PC and ERC). When in doubt, contact the Program Chair.

\section{Prior/Concurrent Submissions}

By submitting a manuscript, the authors guarantee that the manuscript has not been previously published or accepted for publication in a substantially similar form in any conference, journal, or the archived proceedings of a workshop (e.g., in the ACM digital library); but see exceptions below. The authors also guarantee that no paper that contains significant overlap with the contributions of the submitted paper will be under review for any other conference, journal, or workshop having archived proceedings during the review period. Violation of any of these conditions will lead to rejection.

The only exceptions to the above rules are for the authors' own papers in (1) workshops without archived proceedings such as in the ACM digital library (or where the authors chose not to have their paper appear in the archived proceedings), or (2) venues such as IEEE CAL where there is an explicit policy that such publication does not preclude longer conference submissions. In all such cases, the submitted manuscript may ignore the above work to preserve author anonymity. This information must, however, be provided to the Program Chair who  will make this information available to reviewers if it becomes necessary to ensure a fair review. As always, if you are in doubt, it is best to contact the Program Chair. 

Finally, we also note that the ACM Plagiarism Policy covers a range of ethical issues concerning the misrepresentation of other works or one's own work; please consult it carefully.